% Options for packages loaded elsewhere
\PassOptionsToPackage{unicode}{hyperref}
\PassOptionsToPackage{hyphens}{url}
%
\documentclass[
]{article}
\usepackage{amsmath,amssymb}
\usepackage{lmodern}
\usepackage{iftex}
\ifPDFTeX
  \usepackage[T1]{fontenc}
  \usepackage[utf8]{inputenc}
  \usepackage{textcomp} % provide euro and other symbols
\else % if luatex or xetex
  \usepackage{unicode-math}
  \defaultfontfeatures{Scale=MatchLowercase}
  \defaultfontfeatures[\rmfamily]{Ligatures=TeX,Scale=1}
\fi
% Use upquote if available, for straight quotes in verbatim environments
\IfFileExists{upquote.sty}{\usepackage{upquote}}{}
\IfFileExists{microtype.sty}{% use microtype if available
  \usepackage[]{microtype}
  \UseMicrotypeSet[protrusion]{basicmath} % disable protrusion for tt fonts
}{}
\makeatletter
\@ifundefined{KOMAClassName}{% if non-KOMA class
  \IfFileExists{parskip.sty}{%
    \usepackage{parskip}
  }{% else
    \setlength{\parindent}{0pt}
    \setlength{\parskip}{6pt plus 2pt minus 1pt}}
}{% if KOMA class
  \KOMAoptions{parskip=half}}
\makeatother
\usepackage{xcolor}
\usepackage[margin=1in]{geometry}
\usepackage{graphicx}
\makeatletter
\def\maxwidth{\ifdim\Gin@nat@width>\linewidth\linewidth\else\Gin@nat@width\fi}
\def\maxheight{\ifdim\Gin@nat@height>\textheight\textheight\else\Gin@nat@height\fi}
\makeatother
% Scale images if necessary, so that they will not overflow the page
% margins by default, and it is still possible to overwrite the defaults
% using explicit options in \includegraphics[width, height, ...]{}
\setkeys{Gin}{width=\maxwidth,height=\maxheight,keepaspectratio}
% Set default figure placement to htbp
\makeatletter
\def\fps@figure{htbp}
\makeatother
\setlength{\emergencystretch}{3em} % prevent overfull lines
\providecommand{\tightlist}{%
  \setlength{\itemsep}{0pt}\setlength{\parskip}{0pt}}
\setcounter{secnumdepth}{-\maxdimen} % remove section numbering
\usepackage{booktabs}
\usepackage{longtable}
\usepackage{array}
\usepackage{multirow}
\usepackage{wrapfig}
\usepackage{float}
\usepackage{colortbl}
\usepackage{pdflscape}
\usepackage{tabu}
\usepackage{threeparttable}
\usepackage{threeparttablex}
\usepackage[normalem]{ulem}
\usepackage{makecell}
\usepackage{xcolor}
\ifLuaTeX
  \usepackage{selnolig}  % disable illegal ligatures
\fi
\IfFileExists{bookmark.sty}{\usepackage{bookmark}}{\usepackage{hyperref}}
\IfFileExists{xurl.sty}{\usepackage{xurl}}{} % add URL line breaks if available
\urlstyle{same} % disable monospaced font for URLs
\hypersetup{
  pdftitle={Accidents - Regression methods},
  pdfauthor={candice},
  hidelinks,
  pdfcreator={LaTeX via pandoc}}

\title{Accidents - Regression methods}
\author{candice}
\date{2022-12-02}

\begin{document}
\maketitle

\hypertarget{introduction}{%
\subsection{Introduction}\label{introduction}}

This report describes an analysis of the database ``Accidents'', in
which the data was simulated based on a real dataset of events in French
motorway tunnels. Here, we focus on modelling the accidents.

\hypertarget{exploratory-on-numerical-variables}{%
\subsection{Exploratory on numerical
variables}\label{exploratory-on-numerical-variables}}

First, let's look at the data and try to fit a model to our data
distribution.

\begin{verbatim}
## `stat_bin()` using `bins = 30`. Pick better value with `binwidth`.
\end{verbatim}

\includegraphics{accidents_clean_files/figure-latex/Accidents-1.pdf}

To fit a model, let's look at the correlations between the different
variables and their distributions. This will enable us to understand
better how they are possibly linked.

\begin{verbatim}
## Warning in par(usr): argument 1 does not name a graphical parameter

## Warning in par(usr): argument 1 does not name a graphical parameter

## Warning in par(usr): argument 1 does not name a graphical parameter

## Warning in par(usr): argument 1 does not name a graphical parameter

## Warning in par(usr): argument 1 does not name a graphical parameter

## Warning in par(usr): argument 1 does not name a graphical parameter

## Warning in par(usr): argument 1 does not name a graphical parameter

## Warning in par(usr): argument 1 does not name a graphical parameter

## Warning in par(usr): argument 1 does not name a graphical parameter

## Warning in par(usr): argument 1 does not name a graphical parameter

## Warning in par(usr): argument 1 does not name a graphical parameter

## Warning in par(usr): argument 1 does not name a graphical parameter

## Warning in par(usr): argument 1 does not name a graphical parameter

## Warning in par(usr): argument 1 does not name a graphical parameter

## Warning in par(usr): argument 1 does not name a graphical parameter

## Warning in par(usr): argument 1 does not name a graphical parameter

## Warning in par(usr): argument 1 does not name a graphical parameter

## Warning in par(usr): argument 1 does not name a graphical parameter

## Warning in par(usr): argument 1 does not name a graphical parameter

## Warning in par(usr): argument 1 does not name a graphical parameter

## Warning in par(usr): argument 1 does not name a graphical parameter

## Warning in par(usr): argument 1 does not name a graphical parameter

## Warning in par(usr): argument 1 does not name a graphical parameter

## Warning in par(usr): argument 1 does not name a graphical parameter

## Warning in par(usr): argument 1 does not name a graphical parameter

## Warning in par(usr): argument 1 does not name a graphical parameter

## Warning in par(usr): argument 1 does not name a graphical parameter

## Warning in par(usr): argument 1 does not name a graphical parameter

## Warning in par(usr): argument 1 does not name a graphical parameter

## Warning in par(usr): argument 1 does not name a graphical parameter

## Warning in par(usr): argument 1 does not name a graphical parameter

## Warning in par(usr): argument 1 does not name a graphical parameter

## Warning in par(usr): argument 1 does not name a graphical parameter

## Warning in par(usr): argument 1 does not name a graphical parameter

## Warning in par(usr): argument 1 does not name a graphical parameter

## Warning in par(usr): argument 1 does not name a graphical parameter
\end{verbatim}

\includegraphics{accidents_clean_files/figure-latex/unnamed-chunk-1-1.pdf}

It looks like the Accidents are mostly correlated with Lanes and Traffic
but also HGV and Limit. However Lanes is correlated with Length, Traffic
and HGV.\\
The values of Traffic are quite big so we perform a linear regression of
Accidents on Traffic.

\begin{verbatim}
## 
## Call:
## lm(formula = Acc ~ log(Traffic), data = Accidents)
## 
## Residuals:
##    Min     1Q Median     3Q    Max 
## -9.720 -3.402 -1.351  1.173 73.149 
## 
## Coefficients:
##              Estimate Std. Error t value Pr(>|t|)    
## (Intercept)  -42.8780     2.5839  -16.59   <2e-16 ***
## log(Traffic)   3.0430     0.1666   18.26   <2e-16 ***
## ---
## Signif. codes:  0 '***' 0.001 '**' 0.01 '*' 0.05 '.' 0.1 ' ' 1
## 
## Residual standard error: 6.678 on 1158 degrees of freedom
## Multiple R-squared:  0.2236, Adjusted R-squared:  0.223 
## F-statistic: 333.5 on 1 and 1158 DF,  p-value: < 2.2e-16
\end{verbatim}

\includegraphics{accidents_clean_files/figure-latex/linear regression-1.pdf}
\includegraphics{accidents_clean_files/figure-latex/linear regression-2.pdf}

The R squared shows that Traffic is not sufficient to explain the number
of accidents, but the p-value indicates that Traffic is significative.
The residuals are not homoscedastic, nor normally distributed.

\hypertarget{exploratory-on-categorical-variables}{%
\subsection{Exploratory on categorical
variables}\label{exploratory-on-categorical-variables}}

Let's look at lanes. It seems like having two lanes is the case that
creates most accidents. Below, I fit a linear regression.

\begin{verbatim}
## `summarise()` has grouped output by 'Lanes'. You can override using the
## `.groups` argument.
\end{verbatim}

\includegraphics{accidents_clean_files/figure-latex/lanes-1.pdf}
\includegraphics{accidents_clean_files/figure-latex/lanes-2.pdf} Let's
look at year. There is a net increase of accidents in 2007 compared to
before. It still increases in 2008 and then stabilizes.

\begin{verbatim}
## `summarise()` has grouped output by 'Year'. You can override using the
## `.groups` argument.
\end{verbatim}

\includegraphics{accidents_clean_files/figure-latex/year-1.pdf}

Let's look at the Urban variable. We can see that most accidents occurr
in an urban environment.

\begin{verbatim}
## `summarise()` has grouped output by 'Urban'. You can override using the
## `.groups` argument.
\end{verbatim}

\includegraphics{accidents_clean_files/figure-latex/urban-1.pdf}

Let's look at the slope types. There is missing data which will probably
need some further analysis.

\begin{verbatim}
## `summarise()` has grouped output by 'SlopeType'. You can override using the
## `.groups` argument.
\end{verbatim}

\includegraphics{accidents_clean_files/figure-latex/slope type-1.pdf}

Finally, by looking at the companies, one can see an enormous
heterogeneity.

\begin{verbatim}
## `summarise()` has grouped output by 'Tunnel'. You can override using the
## `.groups` argument.
\end{verbatim}

\includegraphics{accidents_clean_files/figure-latex/companies-1.pdf}

Conclusions of our exploratory analysis :\\
\emph{It would probably make sense to consider tunnel and companies as
random }It would maybe be needed to consider two different models
depending on the years we are considering *Numerical variables that look
the most helpful to build the model are Traffic, Lanes, HGV and Limit

\hypertarget{model}{%
\subsection{Model}\label{model}}

\end{document}
